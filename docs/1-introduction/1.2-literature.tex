@c Section 1.2: Literature Review
@node Literature Review
@section Literature Review

Continuous Integration/Continuous Delivery (CI/CD) has revolutionized software development and deployment processes, enabling faster and more efficient delivery of applications. Also, different approaches and workflows, such as GitOps, have been gaining visibility and relevance over time; traditional workflows and practices have evolved to accommodate the changing needs of software development teams. In this literature review are analyzed key publications to gain insights into the background and legacy workflows that modern solutions aim to replace.

The paper published by @i{Beetz and Harrer (@quotedblleft{}GitOps: The Evolution of DevOps?@quotedblright{})}@sup{(@ref{[14],,14})} examines the relationship between DevOps and GitOps, defining these concepts and comparing their key elements, to finally question whether GitOps truly represents the evolution of DevOps or they are still compatible between them.

@noindent @i{Gupta et al. (@quotedblleft{}Prevalence of GitOps, DevOps in Fast CI/CD Cycles@quotedblright{})}@sup{(@ref{[15],,15})} discusses the benefits of GitOps practices in the Kubernetes environment, emphasizing the need for faster and more frequent delivery of software and applications. The authors focus on the implementation of Kubernetes GitOps on AWS, for a true cloud-native development showcase.

@noindent Furthermore, in the context of Agile-based CI/CD projects, @i{Arachchi and Perera (@quotedblleft{}Continuous Integration and Continuous Delivery Pipeline Automation for Agile Software Project Management@quotedblright{})}@sup{(@ref{[16],,16})} present an extended CI/CD pipeline approach, introducing multiple automation phases, such as benchmark, load test, scaling, etc., in order to minimize system interruption and enhance performance.

As we have seen and analyzed, the current state of the art reflects the evolutionary stream of GitOps from traditional DevOps, showcasting the unique characteristics of it and its potential as an evolutionary step in DevOps practices. Moreover, the prevalence and benefits of GitOps in fast CI/CD cycles are discussed, emphasizing its agile and efficient approach to software development. Additionally, the extended CICD pipeline approach presented in the research showcases the integration of benchmarking, load testing, and scaling, contributing to improved project management in Agile environments.
