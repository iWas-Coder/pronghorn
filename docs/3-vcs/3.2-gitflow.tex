@c Section 3.2: Git Workflow Environments
@node Git Workflow Environments
@section Git Workflow Environments
@menu
* Temporary ENV (PR-based)
* Development ENV
* Preproduction ENV
* Production ENV
@end menu

Working with Git should be done in an organised and responsible way, as @i{Linus Torvalds} originally intended. The idea under a continuous development, integration and deployment (CI/CD) pipeline is that, depending on the actions performed on specific branches of the code repository, different actions will be executed according to the time and state of the product.

The two workflows that will be discussed in this paper are: @b{Gitflow}, created and published by @i{Vincent Driessen} back in 2010@sup{(@ref{[4],,4})}, and @b{Trunk Based Development (TBD)}, first created by @i{Paul Hammant} in 2017@sup{(@ref{[7],,7})}, and later extended with the help of multiple collaborators around the world.

With @b{@i{Gitflow}}, we have different types or classes of branches, each of them having a very specific role and interaction policy with the rest. Instead of using a single core/trunk branch, like the original Git design, it uses two main branches, @samp{master} (production) and @samp{develop} (development), each of which defines and describes a different environment. All new features and changes to the source code are implemented in @samp{feature-*} branches, which are forked from and merged to the @samp{develop} branch. When a set of new features is ready for release, a @samp{release-*} branch is forked from the @samp{develop} branch, it only can implement small bug fixes and documentation additions and changes, and is merged to both @samp{master} (as a new production version) and @samp{develop} (as a synchronization method between the development and the production versions) branches. Last but not least, we have the @samp{hotfix-*} branches, which are intended to quickly fix issues in production releases, and are forked from @samp{master} and merged to both @samp{master} and @samp{develop} branches; they can be considered naturally analogous branches to the @samp{release-*} ones, having as counterpoints their existance rationale (@b{user bugs} vs. @b{developer bugs}) and their fork point (@samp{master} vs. @samp{develop}).

@float Figure,gitflow
@shortcaption{Gitflow diagram}
@caption{Gitflow diagram; @pxref{[4],,4}}
@center @image{img/gitflow,7.7cm,,Gitflow diagram,}
@end float

@page

Trunk Based Development (TBD) ...

@page

@include docs/3-vcs/3.2.1-temporary.tex
@include docs/3-vcs/3.2.2-development.tex
@include docs/3-vcs/3.2.3-preproduction.tex
@include docs/3-vcs/3.2.4-production.tex
