@c Section 3.3: GitOps and Repository Segmentation
@node GitOps and Repository Segmentation
@section GitOps and Repository Segmentation

GitOps is a group of operational practices that leverages @b{Git as the single source of truth} for managing and controlling the entire software delivery lifecycle, involving the usage of declarative definitions and version control systems (Git) to manage infrastructure, configuration, and application deployments.

Instead of letting the CI/CD pipeline, with its imperative nature, take care of everything from its position, GitOps practices aim to have application deployment done using a declarative approach.

This is the simplified workflow resulting from the emancipation of the CD from the CI pipeline (for more information, @pxref{Continuous Deployment (CD)}).

@float Figure,gitops
@shortcaption{GitOps toolchain diagram}
@caption{GitOps toolchain diagram}
@center @image{img/gitops,12cm,,GitOps toolchain diagram,}
@end float

@*

Using this approach, our codebase will be modularized and separated in different Git repositories, the essential ones being:

@itemize @bullet
@item Application source code
@item Helm chart source code
@item Kubernetes & ArgoCD Manifests (Configuration)
@end itemize
