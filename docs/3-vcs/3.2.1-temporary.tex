@c Subsection 3.2.1: Temporary (PR-based)
@node Temporary (PR-based)
@subsection Temporary (PR-based)

Depending on the type of workflow we choose, we will tend to orient our project towards a very bifurcated and complex architecture, or towards a simpler and more straightforward solution. In both cases, when the time comes to implement a new feature in the project, it is always a good practice to open a @b{Pull Request (PR)} to, at the very least, let the other collaborators know what features are being developed and give feedback, or simply take a look at them out of pure curiosity. These PRs are just an intermediate step to carry out a merge of two different branches. Apart from their contribution in terms of documentation, they are a vital element when implementing a @b{continuous integration and deployment (CI/CD)} system in our project.

Ultimately, the goal we want to achieve is to be able to execute certain actions depending on the source branch and the target branch specified in the PR definition. A clear example of this would be that, as a developer synchronises new commits in a PR defined as:

@float Figure,merge-math
@shortcaption{Merge-type commit mathematical aproximation}
@caption{Merge-type commit mathematical aproximation}
@displaymath
f(x, y): merge(x \rightarrow y)
@end displaymath
@displaymath
merge(a \rightarrow b): b[\,] + commit(@&a, @&b)
@end displaymath
@displaymath
@&x \equiv HEAD(x)
@end displaymath
@end float

@*

, then a continuous integration process would monitor each synchronisation and compile, test, analyse and provide a development environment where the state of the project can be seen and tested first hand in real time.
This, while seemingly wildly futuristic, is perfectly feasible and efficient; for more information @pxref{Continuous Integration (CI)}.

when se use a @b{TBD} model, this environment also acts as the development and preproduction ones (i.e. these are omitted); thus, we only rely on a temporary/development environment and a production one (@pxref{Production}).
