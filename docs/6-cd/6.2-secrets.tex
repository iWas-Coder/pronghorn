@c Section 6.2: Secret Sealing
@node Secret Sealing
@section Secret Sealing

Until this point, we have a fully-featured and complete pipeline solution that integrates CI/CD, GitOps, IaC and container orchestration both on-premise and cloud-based. However, there is a specifc type of resource that we have not covered: @b{secrets}.

Traditionally, Kubernetes secrets are stored in plaintext within the cluster, which poses a security risk if unauthorized access occurs. @b{Bitnami Sealed Secrets} addresses this concern by leveraging asymmetric encryption (@i{RSA}) to encrypt the secrets before storing them in the cluster and in the Git repositories (using @i{AES-GCM} mode).

This new @samp{SealedSecret} custom resource is safe to share publicly, and only can be decrypted by the controller within the cluster and recover the original @samp{Secret}. This implementation consists of two key components:

@itemize @bullet
@item In-cluster API controller
@item Local @samp{kubeseal} CLI tool
@end itemize

@cartouche
@noindent Creation of a @samp{Secret} sample for demonstration purposes. This example, formatted as a @i{YAML} file, contains the data @samp{foo: bar}.
@end cartouche

@noindent @code{$ echo -n "bar" | kubectl create secret generic example --from-file=foo=/dev/stdin > example.yaml}

@example
apiVersion: v1
kind: Secret
metadata:
  name: example
  creationTimestamp: null
data:
  foo: YmFy
@end example

@*

As seen, the values for all keys in the data field have to be @b{base64-encoded} strings (by default). This, therefore, does not provide any security, so it is not feasible to manage it from a repository.

@cartouche
@noindent Encryption of the secret using the cluster controller's public key. As well as with the previous @samp{kubectl} command, the @samp{kubeseal} tool uses the default cluster configured on the system that we are running the command on.
@end cartouche

@noindent @code{$ kubeseal < example.yaml > example.enc.yaml}

@example
apiVersion: bitnami.com/v1alpha1
kind: SealedSecret
metadata:
  name: example
  creationTimestamp: null
spec:
  template:
    metadata:
      name: example
      namespace: default
      creationTimestamp: null
  encryptedData:
    foo: <ENCRYPTED-AES-GCM>
@end example

@*

With this simple but effective approach, we have the last component that composes the @i{Pronghorn} pipeline, capable of managing everything through Git repositories in an automated way.
