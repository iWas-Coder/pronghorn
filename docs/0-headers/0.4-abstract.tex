@b{Abstract}@*

Pronghorn is an automated Continuous Integration/Continuous Deployment (CI/CD) pipeline designed to simplify and accelerate software development and deployment processes. The pipeline integrates a Git-based architecture that enables version control and collaboration between team members. It consists of various components, including GitLab for version control, SonarQube for code quality analysis, a container registry for storing and deploying container images, a Helm charts repository for storing Kubernetes applications, Bitnami's Sealed Secrets for encrypted secrets management, and ArgoCD for deployment automation.

The pipeline supports different environments, such as development, pre-production and production. Each environment has a specific purpose, and code is tested and validated before moving to the next environment. Pronghorn also incorporates repository segmentation and GitOps methodologies to provide granular control over the deployment process.

Pronghorn's flowchart illustrates how each component interacts with the others and how code moves through the various stages of the pipeline. The design of the pipeline emphasises simplicity, efficiency and speed, enabling rapid software development and deployment.

With Pronghorn, developers can automate their entire software development and deployment process, from code commit to production deployment. The pipeline's Git-based architecture allows developers to collaborate on code and maintain version control throughout the software development process.

Pronghorn also provides additional benefits such as faster deployment times, improved code quality and increased team productivity. Its modular design allows developers to customise and add components to the pipeline as needed.

Overall, Pronghorn provides a customisable, Git-based and automated CI/CD pipeline that can be adapted to different software development environments and processes.

@page

@b{Resumen}@*

Pronghorn es un canal automatizado de integración continua y despliegue continuo (CI/CD) diseñado para simplificar y acelerar los procesos de desarrollo y despliegue de software. El canal integra una arquitectura basada en Git que permite el control de versiones y la colaboración entre los miembros del equipo. Consta de varios componentes, como GitLab para el control de versiones, SonarQube para el análisis de la calidad del código, un registro de contenedores para almacenar y desplegar imágenes de contenedores, un repositorio de gráficos Helm para almacenar aplicaciones Kubernetes, Sealed Secrets de Bitnami para la gestión de secretos cifrados y ArgoCD para la automatización del despliegue.

El pipeline soporta diferentes entornos, como desarrollo, preproducción y producción. Cada entorno tiene un propósito específico, y el código se prueba y valida antes de pasar al siguiente entorno. Pronghorn también incorpora segmentación de repositorios y metodologías GitOps para proporcionar un control granular sobre el proceso de despliegue.

El diagrama de flujo de Pronghorn ilustra cómo interactúa cada componente con los demás y cómo se mueve el código a través de las distintas etapas de la canalización. El diseño de la canalización hace hincapié en la simplicidad, la eficacia y la velocidad, lo que permite un rápido desarrollo e implantación del software.

Con Pronghorn, los desarrolladores pueden automatizar todo el proceso de desarrollo e implantación de software, desde la confirmación del código hasta la implantación en producción. La arquitectura basada en Git permite a los desarrolladores colaborar en el código y mantener el control de versiones durante todo el proceso de desarrollo de software.

Pronghorn también ofrece ventajas adicionales, como tiempos de despliegue más rápidos, mejor calidad del código y mayor productividad del equipo. Su diseño modular permite a los desarrolladores personalizar y añadir componentes al pipeline según sus necesidades.

En general, Pronghorn proporciona un canal de CI/CD automatizado, personalizable y basado en Git que puede adaptarse a diferentes entornos y procesos de desarrollo de software.

@page

@b{Resum}@*

El resum en Català (...).

@c Blank page
@page
@
@page

@end titlepage
