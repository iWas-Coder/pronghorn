@c Section 4.4: Cloud-based Dependant Infrastructure
@node Cloud-based Dependant Infrastructure
@section Cloud-based Dependant Infrastructure

Cloud infrastructure contrasts with the @i{on-premise} solution (@pxref{Infrastructure as Code (IaC/IaaC/IaaS)}) by renting a collection of virtualized resources, including servers, storage, networks, and services, that are provided by cloud service providers over the Internet.

This is an example of a cloud-based infrastructure desired state using @i{Terraform's HCL} to utilize @i{Amazon's AWS} cloud service, in this case deploying an @i{Elastic Compute Cloud (EC2)} node with 1 vCPU and 1GB RAM (@i{t2.micro}).

@cartouche
@noindent This file (@samp{./secrets/aws.tfvars}) will host all variables that will be used later on to configure and deploy resources. Typically, they contain credentials and/or sensitive information, thus, it must be ignored when committing to a repository.
@end cartouche

@example
aws_access_key = "<ACCESS_KEY>"
aws_secret_key = "<SECRET_KEY>"
aws_key_pair_name = "<KEY_PAIR_NAME>"
@end example

@*

@cartouche
@noindent This file (@samp{./providers/aws/aws.tf}) will configure and establish the connection with the service provider (in this case, @i{AWS}), specifying both the region to connect to as well as the credential information needed to authenticate us.
@end cartouche

@example
variable "aws_access_key" @{
  default = "NOT_FOUND"
  sensitive = true
  type = string
@}
variable "aws_secret_key" @{
  default = "NOT_FOUND"
  sensitive = true
  type = string
@}
provider "aws" @{
  region = "eu-south-2"
  access_key = var.aws_access_key
  secret_key = var.aws_secret_key
@}
@end example

@cartouche
@noindent This file (@samp{./providers/aws/os-ubuntu.tf}) serves as a data holder for certain information that will be used ofter while deploying instances with AWS, in this case (i.e. an OS specific version).
@end cartouche

@example
data "aws_ami" "ubuntu" @{
  most_recent = true

  # Ubuntu Server Jammy 22.04 LTS
  filter @{
    name = "name"
    values = ["ubuntu/images/hvm-ssd/ubuntu-jammy-22.04-amd64-server-*"]
  @}

  # Hardware Virtual Machine
  filter @{
    name = "virtualization-type"
    values = ["hvm"]
  @}

  # Canonical
  owners = ["099720109477"]
@}
@end example

@*

@cartouche
@noindent This file (@samp{./nodes/test-node.tf}) is the one where a test node is defined, as an @i{AWS EC2 t2.micro}  instance in this case.
@end cartouche

@example
variable "aws_key_pair_name" @{
  default = "NOT_FOUND"
  sensitive = true
  type = string
@}

resource "aws_instance" "test-node" @{
  ami = data.aws_ami.ubuntu.id
  instance_type = "t2.micro"
  key_name = var.aws_key_pair_name

  tags = @{
    Name = "test-node"
  @}
@}

output "public_ip" @{
  value = aws_instance.test-node.public_ip
@}
@end example

@*

From this point, all needed infrastructure resources can be described and deployed using Terraform, from a simple web server to an entire Kubernetes cluster, and manage all from within a Git repository.
